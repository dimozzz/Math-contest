{\bf Задача 4.}

{\bf а, б.}

Функция $f \in C^{1}[0, 1]$, рассмотрим производную данной функции.
Из уравнения следует соотношений на производную данной функции: \\
$f'(x) = \frac{1}{2} (f'(\frac{x}{2}) + f'(\frac{x + 1}{2}))$

$f'(x)$ --- непрерывна, следовательно достигает на отрезке минимума и максимума. Пусть
максимум достигается в точке $x_0$.
$f'(x_0) = \frac{1}{2} (f'(\frac{x_0}{2}) + f'(\frac{x_0 + 1}{2}))$, следовательно в точках
$\frac{x_0}{2}$ и $\frac{x_0 + 1}{2}$ значения равны $f'(x_0)$.

Аналогично можно позать, что значения также равны в точках $\frac{x_0}{2}, \frac{x_0}{4}, \frac{x_0}{8}, \ldots$
Заметим, что данная последовательность сходится к $0$, следовательно из неприрывности следует, что в нуле $f'$
достигает максимального значения. аналогично в нуле достигается минимальное значение, слеовательно производная постоянна
на всем отрезке $[0, 1]$. ч.т.д.
